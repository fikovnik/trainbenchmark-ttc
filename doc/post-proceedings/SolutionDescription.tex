%!TEX root = ttc15-train-benchmark-sigma.tex

% \enlargethispage{20mm}

\section{Solution Description}
\label{sec:SolutionDescription}

The solution for this transformation case study consist of a set of queries that check for violations of a number of constrains and repair transformations that fixes them.
\SIGMA provides a dedicated model consistency checking DSL with the ability to provide quick fixes repairing invariant validations.
However, given the benchmark framework used in the case study, we decided to provide a more dedicated support, a dedicated internal DSL, for the given query/repair tasks.
The reason is that
\begin{inparaenum}[(1)]
  \item it allows for an easy comparison between the reference implementation in Java and EMF-IncQuery and
  \item it shows the expressiveness of the language allowing one in few lines of code to bridge the gap between problem-level abstractions (query, repair transformation) and implementation-level concepts (\Eg, classes, higher-order functions).
\end{inparaenum}

The solution rely on the \SIGMA common infrastructure which provides a set of basic operations for model navigation (\Ie projecting information from models) and modification (\Ie changing model properties or elements).
The description of the solution is split in two parts:
\begin{inparaenum}[(1)]
  \item the core part that describes the queries and repair transformations,
  \item the integration part gives an overview how it has been integrated in the case study source code.
\end{inparaenum}

\subsection{Queries and Repair Transformations DSL}

Essentially, we have created a small internal DSL that builds on \SIGMA and that allows us to solve the given benchmark cases in an expressive and compact way.
Following the case study description, the top-level domain concept is a \emph{constraint}.
A constraint is composed of a model \emph{query} that finds all model instances violating a certain model restriction and a \emph{repair} transformation correcting the failed instances.
Concretely, a query is a function that given a model element---\Ie a context of the constraint in the classical model consistency checking---returns a set of \emph{matches}.
A match can be either a singe instance or a tuple of instances of model elements that are related to the violations.

\subsection{Constraint DSL}

A typical way of creating an internal DSL in Scala is by designing a library that allows one to write fragments of code with domain-specific syntax.
These fragments are woven within Scala own syntax so that it appears different.

One way to represent the above concepts is using a Scala case class:
%
\begin{scalacode}
case class Constraint[A <: EObject, B <: AnyRef](
  query: (A) => Iterable[B],
  repair: (B) => Unit
)
\end{scalacode}
%
This defines a case class with a field for both query and repair.
A case class in Scala is like a regular class which some additional properties out which, in our case, the important one is that it can be instantiated without the \texttt{new} keyword and thus limiting the language noise.
The two type parameters \texttt{A, B} specify the model context for the query and the types of matches the query produces.
The input type is further constrained to be a subtype of an \texttt{EObject}.
The query and repair are defined as functions \texttt{A} $\rightarrow$ \texttt{Iterable[B]} and \texttt{B} $\rightarrow$ \texttt{Unit} where \texttt{Unit} is like \texttt{void} in Java.

In some cases the match returned by the query is of the same type as the query context.
The query can be therefore simplified to a boolean expression selecting instances on which it evaluates to true.
For these type of queries we provide a dedicated constructs called \texttt{BooleanConstraint}:
%
\begin{scalacode}
case class BooleanConstraint[A <: EObject : ClassTag](
  query: (A) => Boolean, 
  repair: (A) => Unit
)
\end{scalacode}

For example, the first query, \emph{PosLength}, which finds all the segments with negative length can be written as:
%
\begin{scalacode}
BooleanConstraint[Segment](
  query = segment => segment.length < 0,
  repair = segment => segment.length += -segment.length + 1
)
\end{scalacode}
%
In Scala, \texttt{name => ...} is a function literal of one parameter function.
We do not have to specify the types of the parameter nor the result as they will be inferred by the Scala compiler.
% All the constraints are shown in the Appendix~\ref{sec:Constraints}.

Another example using more complex expression is the \emph{SwitchSet} constraint:
%
\begin{scalacode}
Constraint[SwitchPosition, (Semaphore, Route, SwitchPosition, Switch)](
  query = swP => {
    for {
      semaphore <- Option(swP.route.entry) if semaphore.signal == Signal.GO
      sw = swP.switch if sw.currentPosition != swP.position
    } yield (semaphore, swP.route, swP, sw)
  },

  repair = {
    case (_, _, swP, sw) => sw.currentPosition = swP.position
  }
)
\end{scalacode}
%
This is a more complex constraint that matches a tuple of model elements.
It is using a \emph{for comprehension}, a lightweight notation for expressing sequence comprehensions.
From the Scala documentation\footnote{\url{http://docs.scala-lang.org/tutorials/tour/sequence-comprehensions.html}}:
Comprehensions have the form \Scala{for (enumerators) yield e}, where enumerators refers to a semicolon- or new line-separated list of enumerators.
An \emph{enumerator} is either a generator which introduces new variables, or it is a filter.
A comprehension evaluates the body \emph{e} for each binding generated by the enumerators and returns a sequence of these values.

In this concrete example, the generator is the optional value of the \texttt{Route.entry} reference.
It either generates a single value in the case the actual instance contains one or it does not produce anything.
Because, there is a mistake in the model (the \texttt{Route.entry} should have the cardinality set to \texttt{0..1} instead of \texttt{1}), we need to explicitly convert the reference to an \texttt{Option}.

The repair function is defined using a pattern matching construct allowing us to concisely assign variables from the matching tuple.

\subsection{Integration}

The integration consists in making our solution work within the provided benchmark framework.
First, next to constraint syntax, we also need to define its semantics.
For that we define a validator which operationalizes the DSL executing the checks and consequent repairs of the incorrect model instances.
It is defined as an abstract class with two methods that correspond to the two operations:
%
\begin{scalacode}
abstract class Validator[A <: EObject, B <: AnyRef] {
  def check(container: EObject): Iterator[B]
  def repair(matches: Iterator[B]): Unit
}
\end{scalacode}
%

The implementation is straight forward.
For all elements contained in a \texttt{container}, we first collect all instances of the required context type and then query them using the query function provided by the given constraint.
The repair simply executes the constraint repair function on the matching element.

Finally, we instantiate all the constraints, plugs them into the validator and connects the result to the provided framework.
The integration schema is shown in Appendix~\ref{sec:SchemaIntegration}.
We also create a \texttt{SigmaBenchmarkComparator} that is used to compare the matches as required by the case study.
It is a general comparator that either compares single instances (results from boolean constraints violations) or tuples (regular constraints violations).